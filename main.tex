\documentclass{llncs}

% to replace numbering footnotes with the use of symbols
%\usepackage[symbol]{footmisc}
\usepackage{ifpdf}
\usepackage{cite}

\usepackage[pdftex]{graphicx}
\usepackage{sidecap}

\usepackage{enumitem}
\usepackage{booktabs}

\usepackage[cmex10]{amsmath}
\DeclareMathSizes{10}{10}{8.5}{5}

\usepackage{algorithmic}

% *** ALIGNMENT PACKAGES ***
\usepackage{array}
\usepackage[font={footnotesize}]{caption}
\usepackage{subcaption}
\usepackage{mdwmath}
\usepackage{mdwtab}

%\usepackage[tight,footnotesize]{subfigure}
\usepackage{booktabs}
\usepackage{fixltx2e}

\usepackage{url}

\hyphenation{OpenSHMEM PGAS programm-ing}

\usepackage{threeparttable}
\usepackage{tabularx}
\usepackage{array}
\newcolumntype{L}[1]{>{\raggedright\let\newline\\\arraybackslash\hspace{0pt}}m{#1}}
\usepackage{color}
\usepackage[hidelinks]{hyperref}
\usepackage{stfloats} %permits stretching and shrinking images
\usepackage{listings}
\usepackage{afterpage}

\usepackage{framed}  %package for framing images
\usepackage{lscape}
%\usepackage{flushend}

\lstset{ 
  language=C,           % the language of the code
  basicstyle=\tiny,       % the size of the fonts that are used for the code
%  numbers=left,                   % where to put the line-numbers
%  numberstyle=\tiny\color{gray},  % the style that is used for the line-numbers
  stepnumber=1,                   % the step between two line-numbers. If it's 1, each line 
                                    % will be numbered 
  backgroundcolor=\color{white},  % choose the background color. You must add \usepackage{color}
  showspaces=false,               % show spaces adding particular underscores
  showstringspaces=false,         % underline spaces within strings
  showtabs=false,                 % show tabs within strings adding particular underscores
  rulecolor=\color{black},        % if not set, the frame-color may be changed on line-breaks within not-black text (e.g. commens (green here))
  tabsize=1,                      % sets default tabsize to 2 spaces
  captionpos=b,                   % sets the caption-position to bottom
  breaklines=true,                % sets automatic line breaking
  breakatwhitespace=true,        % sets if automatic breaks should only happen at whitespace
  title=\lstname,                 % show the filename of files included with \lstinputlisting;
                                  % also try caption instead of title
%  keywordstyle=\color{blue},          % keyword style
%  commentstyle=\color{dkgreen},       % comment style
% stringstyle=\color{mauve},         % string literal style
  escapeinside={\%*}{*)},            % if you want to add LaTeX within your code
%  xleftmargin=5pt,
  morekeywords={*,...}              % if you want to add more keywords to the set
}

\usepackage{wrapfig,booktabs}


\newcommand{\myfigureshrinker}{\vspace{-1cm}}
%\usepackage[font={small}]{caption, subfig}
\setlength{\abovecaptionskip}{1ex}
\setlength{\belowcaptionskip}{1ex}
\setlength{\floatsep}{1ex}
\setlength{\textfloatsep}{1ex}

%\usepackage[sort&compress]{natbib}    
\newcommand{\bibfont}{\small}
%\setlength{\bibsep}{0ex}


\setlength{\parskip}{0pt}
\setlength{\parsep}{0pt}
\setlength{\headsep}{0pt}
\setlength{\topskip}{0pt}
\setlength{\topmargin}{0pt}
\setlength{\topsep}{0pt}
\setlength{\partopsep}{0pt}
\linespread{0.95}


% DONOT Use with LNCS: [reduce space between
% paragraphs]
%\usepackage[compact]{titlesec}
%\titlespacing{\section}{0pt}{*0}{*0}
%\titlespacing{\subsection}{0pt}{*0}{*0}
%\titlespacing{\subsubsection}{0pt}{*0}{*0}

\makeatletter
\renewcommand\subsubsection{\@startsection{subsubsection}{3}{\z@}%
                       {-8\p@ \@plus -4\p@ \@minus -4\p@}% Formerly -18\p@ \@plus -4\p@ \@minus -4\p@
                       {-0.5em \@plus -0.22em \@minus -0.1em}%
                       {\normalfont\normalsize\bfseries\boldmath}}
\makeatother



\begin{document}

\title{Active Messages for OpenSHMEM}

\author{Siddhartha Jana}
\institute{ 
Department of Computer Science,\\ University of Houston,\\ Houston, Texas\\
\email{sidjana@cs.uh.edu}
}

\date{14th March, 2016}
\maketitle

\begin{abstract}
    \label{sec:abstract}
    
This document introduces the concept of Active 
Messages within OpenSHMEM. The Active Message
interface provides a medium for a PE to launch 
tasks on a remote PE. These tasks may be used for 
performing computation or data transfers on the 
remote PE.


\end{abstract}

% Write about GASNet and MPI proposals. Also write 
% about past Active Message work from the 1994 
% paper
\section{Introduction and Related Work}
\label{sec:intro}

There exists multiple communication libraries that 
support active messages. Examples include GASNet, 
IBM's LAPI and DCMF, Myrinet's MX, POOMA/CHEETAH 
as well as application layers like PBGL. There 
have also been proposals of introducing this 
concept into MPI. 



% write about different terms like endpoint - 
% refer gasnet spec
\section{Terminology}
\label{sec:terminology}

% write the description of the new function 
% interfaces being proposed
\section{Proposed API} 
\label{sec:api}

\subsection{Registraction of Active Message Handlers}
\subsubsection{shmemx\_am\_attach}
Enables the calling PE to register a handler with 
the OpenSHMEM implementation. This is a collective 
operation. The prototype of the handler differs 
slightly based on the type of communication model 
being used.\\

\fbox{\begin{minipage}{28em}
\textit{void shmemx\_am\_attach (int handler\_id, 
shmemx\_am\_handler function\_handler)}
\end{minipage}}

\begin{itemize}
    \item \textit{handler\_id} The integer Id used 
        to identify an AM handler
    \item \textit{function\_handler} A pointer to 
        the function that holds the body of the 
        AM handler. The signature of 
        the function handler is:
        \begin{itemize}
            \item For 1-sided Active 
                Messages:\\\textit{void 
                function\_name (void *buf, size\_t 
            nbytes, int req\_pe)}
            \item For 2-sided Active 
                Messages:\\\textit{void 
                function\_name (void *buf, size\_t nbytes, int req\_pe, shmemx\_am\_token\_t token)}
            \item Where,
                \begin{itemize}
                    \item \textit{*buf} The 
                        pointer to the user buffer 
                        being transferred to the 
                        remote PE 
                        \textit{req\_pe}.
                    \item \textit{nbytes} Size of 
                        the user buffer 
                        \textit{buf}
                    \item \textit{req\_pe} Id of 
                        the remote PE that will 
                        execute the handler 
                \end{itemize}
       \end{itemize}
\end{itemize}

\subsubsection{shmemx\_am\_detach}
A call to this function removes the mapping 
between the handler id and the function. This is a 
collective operation. Once detached, it is illegal 
for a PE to initiate an active message with the 
same function handler id unless it explicitly maps 
the id again using shmemx\_am\_attach.\\

\fbox{\begin{minipage}{28em}
\textit{void shmemx\_am\_detach (int 
handler\_id);}
\end{minipage}}

\begin{itemize}
    \item \textit{handler\_id} This is the handler 
        id of the function that needs to be 
        deregistered with the underlying 
        implementation.
\end{itemize}


\subsection{Initiating Active Messages}

\subsubsection{shmemx\_am\_launch}
This function is used to launch an active message
on a remote PE. There is no guarantee of 
completion of execution of the handler on function 
return. The source buffer can be reused on 
return.\\

\fbox{\begin{minipage}{28em}
void shmemx\_am\_launch (int dest, int handler\_id, void* source\_addr, size\_t nbytes)
\end{minipage}}


 \begin{itemize}
     \item \textit{dest} Id of the remote PE that 
         will execute the AM handler
     \item \textit{handler\_id} Id of the handler 
         that is executed at the remote PE.
     \item \textit{source\_addr} Start address of 
         the user buffer that is passed to the 
         AM handler
     \item \textit{nbytes} Size of the user buffer 
         \textit{source\_addr}
 \end{itemize}

\subsubsection{shmemx\_am\_request}
In a two-sided request-reply communication model, 
this function is used to launch a request AM handler on 
the remote PE. Typically, the request handler in 
turn initiates a reply AM handler on 
the current PE.\\
  
\fbox{\begin{minipage}{28em}
void shmemx\_am\_request(int dest, int handler\_id, void* source\_addr, size\_t nbytes)
\end{minipage}}


 \begin{itemize}
     \item \textit{dest} Id of the remote PE that 
         will execute the AM handler
     \item \textit{handler\_id} Id of the handler 
         that is executed at the remote PE.
     \item \textit{source\_addr} Start address of 
         the user buffer that is passed to the 
         AM handler
     \item \textit{nbytes} Size of the user buffer 
         \textit{source\_addr}
 \end{itemize}

\subsubsection{shmemx\_am\_reply}

        In a two-sided request-reply communication 
        model, this function is used by the 
        request
        AM handler to launch a reply AM handler at 
        the remote PE that had originally 
        initiated the executing handler on 
        the current PE.\\

\fbox{\begin{minipage}{28em}
void shmemx\_am\_reply(int handler\_id, void* source\_addr, size\_t nbytes, shmemx\_am\_token\_t temp\_token)
\end{minipage}}

\begin{itemize}
    \item \textit{handler\_id} Id of the handler 
        that is executed at the remote PE.
    \item \textit{source\_addr} Start address of 
        the user buffer that is passed to the AM 
        handler
    \item \textit{nbytes} Size of the user buffer 
        \textit{source\_addr}
\item {temp\_token} This is the token passed to 
     the executing request handler. This token is 
     passed as is to the reply handler.
\end{itemize}

\subsection{Completion of Active Messages}

\subsubsection{shmemx\_am\_quiet}
This function waits till the completion of all 
active messages initiated by the current PE.\\

\fbox{\begin{minipage}{12em}
void shmemx\_am\_quiet()
\end{minipage}}



% write about impact of synchronization, etc and 
% its impact on active msgs
\section{Impact on existing OpenSHMEM 
implementation}
\label{sec:impact}

% write about further proposals (not in 
% implementation) viz. active message broadcast, 
% active sets
\section{Possible Extensions}
\label{sec:extensions}

% Point to the location of the github repo
\section{Microbenchmarks}
\label{sec:microbench}

\section{Conclusion}
\label{sec:conclusion}

\section{Acknowledgments}
The author is to be solely blamed for this report. 
This report is not exhaustive and does NOT take 
into account the challenges that arise in 
introducing the concept of Active Messages in all 
architectures that support OpenSHMEM. There are 
some important lessons to be learned from the 
GASNet community.

% This work is supported by the joint collaboration 
% between the Las Alamos National Laboratory and the 
% University of Houston.

\bibliographystyle{splncs03}
\bibliography{files/references}

\end{document}

