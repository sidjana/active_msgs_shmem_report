
\subsection{Registraction of Active Message Handlers}
\subsubsection{shmemx\_am\_attach}
Enables the calling PE to register a handler with 
the OpenSHMEM implementation. This is a collective 
operation. The prototype of the handler differs 
slightly based on the type of communication model 
being used.\\

\fbox{\begin{minipage}{28em}
\textit{void shmemx\_am\_attach (int handler\_id, 
shmemx\_am\_handler function\_handler)}
\end{minipage}}

\begin{itemize}
    \item \textit{handler\_id} The integer Id used 
        to identify an AM handler
    \item \textit{function\_handler} A pointer to 
        the function that holds the body of the 
        AM handler. The signature of 
        the function handler is:
        \begin{itemize}
            \item For 1-sided Active 
                Messages:\\\textit{void 
                function\_name (void *buf, size\_t 
            nbytes, int req\_pe)}
            \item For 2-sided Active 
                Messages:\\\textit{void 
                function\_name (void *buf, size\_t nbytes, int req\_pe, shmemx\_am\_token\_t token)}
            \item Where,
                \begin{itemize}
                    \item \textit{*buf} The 
                        pointer to the user buffer 
                        being transferred to the 
                        remote PE 
                        \textit{req\_pe}.
                    \item \textit{nbytes} Size of 
                        the user buffer 
                        \textit{buf}
                    \item \textit{req\_pe} Id of 
                        the remote PE that will 
                        execute the handler 
                \end{itemize}
       \end{itemize}
\end{itemize}

\subsubsection{shmemx\_am\_detach}
A call to this function removes the mapping 
between the handler id and the function. This is a 
collective operation. Once detached, it is illegal 
for a PE to initiate an active message with the 
same function handler id unless it explicitly maps 
the id again using shmemx\_am\_attach.\\

\fbox{\begin{minipage}{28em}
\textit{void shmemx\_am\_detach (int 
handler\_id);}
\end{minipage}}

\begin{itemize}
    \item \textit{handler\_id} This is the handler 
        id of the function that needs to be 
        deregistered with the underlying 
        implementation.
\end{itemize}


\subsection{Initiating Active Messages}

\subsubsection{shmemx\_am\_launch}
This function is used to launch an active message
on a remote PE. There is no guarantee of 
completion of execution of the handler on function 
return. The source buffer can be reused on 
return.\\

\fbox{\begin{minipage}{28em}
void shmemx\_am\_launch (int dest, int handler\_id, void* source\_addr, size\_t nbytes)
\end{minipage}}


 \begin{itemize}
     \item \textit{dest} Id of the remote PE that 
         will execute the AM handler
     \item \textit{handler\_id} Id of the handler 
         that is executed at the remote PE.
     \item \textit{source\_addr} Start address of 
         the user buffer that is passed to the 
         AM handler
     \item \textit{nbytes} Size of the user buffer 
         \textit{source\_addr}
 \end{itemize}

\subsubsection{shmemx\_am\_request}
In a two-sided request-reply communication model, 
this function is used to launch a request AM handler on 
the remote PE. Typically, the request handler in 
turn initiates a reply AM handler on 
the current PE.\\
  
\fbox{\begin{minipage}{28em}
void shmemx\_am\_request(int dest, int handler\_id, void* source\_addr, size\_t nbytes)
\end{minipage}}


 \begin{itemize}
     \item \textit{dest} Id of the remote PE that 
         will execute the AM handler
     \item \textit{handler\_id} Id of the handler 
         that is executed at the remote PE.
     \item \textit{source\_addr} Start address of 
         the user buffer that is passed to the 
         AM handler
     \item \textit{nbytes} Size of the user buffer 
         \textit{source\_addr}
 \end{itemize}

\subsubsection{shmemx\_am\_reply}

        In a two-sided request-reply communication 
        model, this function is used by the 
        request
        AM handler to launch a reply AM handler at 
        the remote PE that had originally 
        initiated the executing handler on 
        the current PE.\\

\fbox{\begin{minipage}{28em}
void shmemx\_am\_reply(int handler\_id, void* source\_addr, size\_t nbytes, shmemx\_am\_token\_t temp\_token)
\end{minipage}}

\begin{itemize}
    \item \textit{handler\_id} Id of the handler 
        that is executed at the remote PE.
    \item \textit{source\_addr} Start address of 
        the user buffer that is passed to the AM 
        handler
    \item \textit{nbytes} Size of the user buffer 
        \textit{source\_addr}
\item {temp\_token} This is the token passed to 
     the executing request handler. This token is 
     passed as is to the reply handler.
\end{itemize}

\subsection{Completion of Active Messages}

\subsubsection{shmemx\_am\_quiet}
This function waits till the completion of all 
active messages initiated by the current PE.\\

\fbox{\begin{minipage}{12em}
void shmemx\_am\_quiet()
\end{minipage}}

