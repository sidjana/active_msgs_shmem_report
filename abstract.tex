\documentclass{llncs}

% to replace numbering footnotes with the use of symbols
%\usepackage[symbol]{footmisc}
\usepackage{ifpdf}
\usepackage{cite}

\usepackage[pdftex]{graphicx}

\usepackage{enumitem}
\usepackage{booktabs}

\usepackage[cmex10]{amsmath}


\usepackage{algorithmic}

% *** ALIGNMENT PACKAGES ***
\usepackage{array}
\usepackage[font={footnotesize}]{caption}
\usepackage{subcaption}
\usepackage{mdwmath}
\usepackage{mdwtab}

%\usepackage[tight,footnotesize]{subfigure}
\usepackage{booktabs}
\usepackage{fixltx2e}

\usepackage{url}

\hyphenation{ RTS CTS OpenSHMEM Shar-ed Memo-ry Pass-ing  Mess-age Be-sides bench-markmicro-benchmark uni-versity}

\usepackage{threeparttable}
\usepackage{tabularx}
\usepackage{array}
\newcolumntype{L}[1]{>{\raggedright\let\newline\\\arraybackslash\hspace{0pt}}m{#1}}
\usepackage{color}
\usepackage[hidelinks]{hyperref}
\usepackage{stfloats} %permits stretching and shrinking images
\usepackage{listings}
\usepackage{afterpage}

\usepackage{framed}  %package for framing images
\usepackage{lscape}
%\usepackage{flushend}

\lstset{ 
  language=C,           % the language of the code
  basicstyle=\tiny,       % the size of the fonts that are used for the code
%  numbers=left,                   % where to put the line-numbers
%  numberstyle=\tiny\color{gray},  % the style that is used for the line-numbers
  stepnumber=1,                   % the step between two line-numbers. If it's 1, each line 
                                    % will be numbered 
  backgroundcolor=\color{white},  % choose the background color. You must add \usepackage{color}
  showspaces=false,               % show spaces adding particular underscores
  showstringspaces=false,         % underline spaces within strings
  showtabs=false,                 % show tabs within strings adding particular underscores
  rulecolor=\color{black},        % if not set, the frame-color may be changed on line-breaks within not-black text (e.g. commens (green here))
  tabsize=1,                      % sets default tabsize to 2 spaces
  captionpos=b,                   % sets the caption-position to bottom
  breaklines=true,                % sets automatic line breaking
  breakatwhitespace=true,        % sets if automatic breaks should only happen at whitespace
  title=\lstname,                 % show the filename of files included with \lstinputlisting;
                                  % also try caption instead of title
%  keywordstyle=\color{blue},          % keyword style
%  commentstyle=\color{dkgreen},       % comment style
% stringstyle=\color{mauve},         % string literal style
  escapeinside={\%*}{*)},            % if you want to add LaTeX within your code
%  xleftmargin=5pt,
  morekeywords={*,...}              % if you want to add more keywords to the set
}



\setlength{\parskip}{0pt}
\setlength{\parsep}{0pt}
\setlength{\headsep}{0pt}
\setlength{\topskip}{0pt}
\setlength{\topmargin}{0pt}
\setlength{\topsep}{0pt}
\setlength{\partopsep}{0pt}
%\linespread{1}

% DONOT Use with LNCS: [reduce space between
% paragraphs]
%\usepackage[compact]{titlesec}
%\titlespacing{\section}{0pt}{*0}{*0}
%\titlespacing{\subsection}{0pt}{*0}{*0}
%\titlespacing{\subsubsection}{0pt}{*0}{*0}



\begin{document}

\title{Impact of Design of Communication Protocols and
Choice of Transport Layer on the Power and Energy
Consumption of Distributed Programming Models}

\author{ Siddhartha Jana\inst{1} \and  Oscar Hernandez\inst{2}  \and Stephen Poole\inst{2} \and Barbara Chapman\inst{1}}
\institute{ 
Computer Science department,\\ University of Houston,\\ Houston, Texas\\
\email{{sidjana,chapman}@cs.uh.edu}
\and
Computer Science and Mathematics Division\\
Oak Ridge National Laboratory,\\
Oak Ridge, Tennessee\\
\email{{oscar,spoole}@ornl.gov}
}

\date{6th February, 2014}
\maketitle

\begin{abstract}

    The amount of energy consumed due to data movement
    poses a serious threat to the usability of
    distributed programming models on future systems.
    Message-passing models like MPI provide the user
    with explicit interfaces to initiate data-transfers
    among remote processes.  In this work, we establish
    the notion that from a programmer's standpoint, the
    controllable factors like the size of the
    data-payload to be transferred and the number of
    explicit MPI calls to service such transfers have a
    direct impact on the pattern of the
    power-signatures of communication kernels.  On a
    closer look, we further observe that the choice of
    the transport layer (along with the associated
    interconnect) and the design of the communication
    protocol to implement these transfer-routines are
    responsible for this pattern.  This paper discusses
    these initial initial attempts in performing a
    fine-grained study on the impact of the power and
    energy consumption due to data movement in
    distributed programming models.  We hope that
    results discussed in this work would motivate the
    design of ``power-aware" middleware for use with
    HPC applications.

\end{abstract}

\end{document}

